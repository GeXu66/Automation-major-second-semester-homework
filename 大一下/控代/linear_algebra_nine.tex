\documentclass{article}
\usepackage{amsmath}
\usepackage{amssymb}
\usepackage{ctex}
\usepackage{graphicx}
\begin{document}
\title{�ھŴ���ҵ}
\maketitle
\[\begin{gathered}
  1.bilinear: \hfill \\
  (\left[ \begin{gathered}
  {x_1} \hfill \\
  {x_2} \hfill \\
\end{gathered}  \right],\left[ \begin{gathered}
  {y_1} \hfill \\
  {y_2} \hfill \\
\end{gathered}  \right] + \left[ \begin{gathered}
  {z_1} \hfill \\
  {z_2} \hfill \\
\end{gathered}  \right]) = (\left[ \begin{gathered}
  {x_1} \hfill \\
  {x_2} \hfill \\
\end{gathered}  \right],\left[ \begin{gathered}
  {y_1} \hfill \\
  {y_2} \hfill \\
\end{gathered}  \right]) + (\left[ \begin{gathered}
  {x_1} \hfill \\
  {x_2} \hfill \\
\end{gathered}  \right],\left[ \begin{gathered}
  {z_1} \hfill \\
  {z_2} \hfill \\
\end{gathered}  \right]) = |{x_1}{y_1}| + |{x_2}{y_2}| + |{x_1}{z_1}| + |{x_2}{z_2}| \hfill \\
  (\left[ \begin{gathered}
  {x_1} \hfill \\
  {x_2} \hfill \\
\end{gathered}  \right],\left[ \begin{gathered}
  {y_1} \hfill \\
  {y_2} \hfill \\
\end{gathered}  \right] + \left[ \begin{gathered}
  {z_1} \hfill \\
  {z_2} \hfill \\
\end{gathered}  \right]) = (\left[ \begin{gathered}
  {x_1} \hfill \\
  {x_2} \hfill \\
\end{gathered}  \right],\left[ \begin{gathered}
  {y_1} + {z_1} \hfill \\
  {y_2} + {z_2} \hfill \\
\end{gathered}  \right]) = |{x_1}({y_1} + {z_1})| + |{x_2}({y_2} + {z_2})| \hfill \\
   = |{x_1}{y_1} + {x_1}{z_1}| + |{x_2}{y_2} + {x_2}{z_2}| \leqslant |{x_1}{y_1}| + |{x_2}{y_2}| + |{x_1}{z_1}| + |{x_2}{z_2}| \hfill \\
  {\text{so it does't satisfy bilinear,it's not an inner product}}{\text{.}} \hfill \\
\end{gathered} \]
\[\begin{gathered}
  2.(1)(u + v,u - v) = (u,u) - (u,v) + (v,u) - (v,v) \hfill \\
   = ||u|{|^2} - ||v|{|^2} \hfill \\
  (2){\text{if u,v have the same norm,then }}||u|| = ||v|| \hfill \\
  (u + v,u - v) = ||u|{|^2} - ||v|{|^2} = 0 \hfill \\
  {\text{so u + v is orthogonal to u - v}} \hfill \\
  {\text{(3)let u and v be two side length of a rhombus,}} \hfill \\
  {\text{then u + v and u - v are two digonals of the rhombus,and the length of u and v are same}}{\text{.}} \hfill \\
  {\text{so }}(u + v,u - v) = ||u|{|^2} - ||v|{|^2} = 0,{\text{so the diagonals of the rhombus are perpendicular}}{\text{.}} \hfill \\
\end{gathered} \]
\[\begin{gathered}
  3.||u|| \leqslant ||u + av|| \Rightarrow ||u|{|^2} \leqslant ||u + av|{|^2} \hfill \\
  0 \leqslant 2a(u,v) + {a^2}{v^2} \hfill \\
  f(a) = {a^2}{v^2} + 2a(u,v) = 0 \hfill \\
  then{\text{ }}{a_1} = 0,{a_2} = \frac{{ - 2(u,v)}}{{||v|{|^2}}} \hfill \\
  {\text{and between }}{a_1}{\text{ and }}{a_2},f(a) < 0 \hfill \\
  {\text{so if }}f(a) \geqslant 0{\text{ to }}\forall {\text{a}} \in {\text{R,then (u,v) = 0}} \hfill \\
\end{gathered} \]
\[\begin{gathered}
  4.\int_{ - \pi }^\pi  {\frac{{\cos nx}}{{\sqrt \pi  }}\frac{{\cos mx}}{{\sqrt \pi  }}dx = \frac{1}{\pi }\int_{ - \pi }^\pi  {\cos nx\cos mxdx = } } \frac{1}{{2\pi }}\int_{ - \pi }^\pi  {\cos ((n + m)x) + \cos ((n - m)x)dx}  \hfill \\
   = \frac{1}{{2\pi }}[\frac{{\sin ((n + m)x)}}{{n + m}} + \frac{{\sin ((n - m)x)}}{{n - m}}]|_{ - \pi }^\pi  = 0 \hfill \\
  \int_{ - \pi }^\pi  {\frac{{\sin nx}}{{\sqrt \pi  }}\frac{{\sin mx}}{{\sqrt \pi  }}dx}  = \frac{1}{\pi }\int_{ - \pi }^\pi  {\sin nx\sin mxdx = }  - \frac{1}{{2\pi }}\int_{ - \pi }^\pi  {\cos ((n + m)x) - \cos ((n - m)x)dx}  \hfill \\
   =  - \frac{1}{{2\pi }}[\frac{{\sin ((n + m)x)}}{{n + m}} - \frac{{\sin ((n - m)x)}}{{n - m}}]|_{ - \pi }^\pi  = 0 \hfill \\
  \int_{ - \pi }^\pi  {\frac{{\sin nx}}{{\sqrt \pi  }}\frac{{\cos mx}}{{\sqrt \pi  }}dx}  = \frac{1}{\pi }\int_{ - \pi }^\pi  {\sin nx\cos mxdx = } 0 \hfill \\
  \int_{ - \pi }^\pi  {\frac{1}{{\sqrt {2\pi } }}} \frac{{\cos nx}}{{\sqrt \pi  }}dx = \frac{1}{{\pi \sqrt 2 }}\int_{ - \pi }^\pi  {\cos nx = 0}  \hfill \\
  \int_{ - \pi }^\pi  {\frac{1}{{\sqrt {2\pi } }}} \frac{{\sin nx}}{{\sqrt \pi  }}dx = \frac{1}{{\pi \sqrt 2 }}\int_{ - \pi }^\pi  {\sin nx = 0}  \hfill \\
  \int_{ - \pi }^\pi  {\frac{{\sin nx}}{{\sqrt \pi  }}\frac{{\sin nx}}{{\sqrt \pi  }}dx}  = \frac{1}{\pi }\int_{ - \pi }^\pi  {{{\sin }^2}nxdx = } 1 \hfill \\
  \int_{ - \pi }^\pi  {\frac{{\cos nx}}{{\sqrt \pi  }}\frac{{\cos nx}}{{\sqrt \pi  }}dx}  = \frac{1}{\pi }\int_{ - \pi }^\pi  {{{\cos }^2}nxdx = } 1 \hfill \\
  \int_{ - \pi }^\pi  {\frac{1}{{\sqrt {2\pi } }}} \frac{1}{{\sqrt {2\pi } }}dx = \frac{1}{{2\pi }}\int_{ - \pi }^\pi  {dx = 1}  \hfill \\
\end{gathered} \]
\[\begin{gathered}
  5.{v_1} = 1,{v_2} = x,{v_3} = {x^2} \hfill \\
  {u_1} = 1 \hfill \\
  {u_2} = {v_2} - \frac{{{v_2} \cdot {u_1}}}{{{u_1} \cdot {u_1}}}{u_1} = x - \frac{{\int_0^1 {xdx} }}{{\int_0^1 {1dx} }} \cdot 1 = x - \frac{1}{2} \hfill \\
  {u_3} = {v_3} - \frac{{{v_3} \cdot {u_1}}}{{{u_1} \cdot {u_1}}}{u_1} - \frac{{{v_3} \cdot {u_2}}}{{{u_2} \cdot {u_2}}}{u_2} = {x^2} - \frac{{\int_0^1 {{x^2}dx} }}{{\int_0^1 {1dx} }} \cdot 1 - \frac{{\int_0^1 {{x^2}(x - \frac{1}{2})dx} }}{{\int_0^1 {{{(x - \frac{1}{2})}^2}dx} }}(x - \frac{1}{2})\\
  = {x^2} - \frac{1}{3} - (x - \frac{1}{2}) = {x^2} - x + \frac{1}{6} \hfill \\
  \int_0^1 {{{(x - \frac{1}{2})}^2} = \frac{1}{{12}} \Rightarrow } \int_0^1 {{{(2\sqrt 3 )}^2}{{(x - \frac{1}{2})}^2} = 1 \Rightarrow } {u_2} = 2\sqrt 3 (x - \frac{1}{2}) \hfill \\
  \int_0^1 {{{({x^2} - x + \frac{1}{6})}^2}}  = \frac{1}{{180}} \Rightarrow {u_3} = \sqrt 5 (6{x^2} - 6x + 1) \hfill \\
  \left\{ \begin{gathered}
  {u_1} = 1 \hfill \\
  {u_2} = 2\sqrt 3 (x - \frac{1}{2}) \hfill \\
  {u_3} = \sqrt 5 (6{x^2} - 6x + 1) \hfill \\
\end{gathered}  \right. \hfill \\
\end{gathered} \]
\[\begin{gathered}
  6.{\text{the orthonormal basis of }}{{\text{P}}^2}{\text{[x] is }} \hfill \\
  \left\{ \begin{gathered}
  {u_1} = 1 \hfill \\
  {u_2} = 2\sqrt 3 (x - \frac{1}{2}) \hfill \\
  {u_3} = \sqrt 5 (6{x^2} - 6x + 1) \hfill \\
\end{gathered}  \right.,q = {k_1}{u_1} + {k_2}{u_2} + {k_3}{u_3} \hfill \\
  {k_1} = (q,{u_1}) = \int_0^1 {q{u_1}dx = {u_1}(\frac{1}{2})}  = 1 \hfill \\
  {k_2} = (q,{u_2}) = \int_0^1 {q{u_2}dx = {u_2}(\frac{1}{2})}  = 0 \hfill \\
  {k_3} = (q,{u_3}) = \int_0^1 {q{u_3}dx = {u_3}(\frac{1}{2})}  =  - \frac{{\sqrt 5 }}{2} \hfill \\
  q = {u_1} - \frac{{\sqrt 5 }}{2}{u_3} = 1 - \frac{5}{2}(6{x^2} - 6x + 1) =  - 15{x^2} + 15x - \frac{3}{2} \hfill \\
\end{gathered} \]
\[\begin{gathered}
  7.{v_1} = (1,2,3, - 4),{v_2} = ( - 5,4,3,2) \hfill \\
  {u_1} = {v_1} \hfill \\
  {u_2} = {v_2} - \frac{{{v_2} \cdot {u_1}}}{{{u_1} \cdot {u_1}}} \cdot {u_1} = ( - \frac{{77}}{{15}},\frac{{56}}{{15}},\frac{{13}}{5},\frac{{38}}{{15}}) \hfill \\
  {x_1} = \frac{{{u_1}}}{{||{u_1}||}} = (\frac{1}{{\sqrt {30} }},\frac{2}{{\sqrt {30} }},\frac{3}{{\sqrt {30} }},\frac{{ - 4}}{{\sqrt {30} }}) \hfill \\
  {x_2} = \frac{{{u_2}}}{{||{u_2}||}} = ( - \frac{{77}}{{15}}\sqrt {\frac{{15}}{{802}}} ,\frac{{56}}{{15}}\sqrt {\frac{{15}}{{802}}} ,\frac{{13}}{5}\sqrt {\frac{{15}}{{802}}} ,\frac{{38}}{{15}}\sqrt {\frac{{15}}{{802}}} ) \hfill \\
  {\text{suppose }}{v_1} \cdot x = 0,{v_2} \cdot x = 0 \hfill \\
  \left\{ \begin{gathered}
  {x_1} + 2{x_2} + 3{x_3} - 4{x_4} = 0 \hfill \\
   - 5{x_1} + 4{x_2} + 3{x_3} + 2{x_4} = 0 \hfill \\
\end{gathered}  \right. \hfill \\
   \Rightarrow (10,9,0,7),(3,9, - 7,0) \hfill \\
  {x_3} = (\frac{{10}}{{\sqrt {230} }},\frac{9}{{\sqrt {230} }},0,\frac{7}{{\sqrt {230} }}) \hfill \\
  {x_4} = (\frac{3}{{\sqrt {139} }},\frac{9}{{\sqrt {139} }},\frac{{ - 7}}{{\sqrt {139} }},0) \hfill \\
  {x_3},{x_4}{\text{ are two orthonormal basis of }}{{\text{U}}^ \bot } \hfill \\
\end{gathered} \]
\[\begin{gathered}
  8.U = span((1,1,0,0),(0,0,1,2)) \hfill \\
  C = \left[ \begin{gathered}
  {\text{1 0}} \hfill \\
  {\text{1 0}} \hfill \\
  {\text{0 1}} \hfill \\
  0{\text{ 2}} \hfill \\
\end{gathered}  \right],{\text{we need to find x = }}\left[ \begin{gathered}
  {x_1} \hfill \\
  {x_2} \hfill \\
\end{gathered}  \right]{\text{ so that Cx = b and b is near }}\left[ \begin{gathered}
  1 \hfill \\
  2 \hfill \\
  3 \hfill \\
  4 \hfill \\
\end{gathered}  \right] \hfill \\
  {C^T}Cx = {C^T}b,x = {({C^T}C)^{ - 1}}{C^T}b \hfill \\
  {\text{so }}u=Cx = C{({C^T}C)^{ - 1}}{C^T}\left[ \begin{gathered}
  1 \hfill \\
  2 \hfill \\
  3 \hfill \\
  4 \hfill \\
\end{gathered}  \right] = \left[ \begin{gathered}
  1.5 \hfill \\
  1.5 \hfill \\
  2.2 \hfill \\
  4.4 \hfill \\
\end{gathered}  \right] \hfill \\
\end{gathered} \]
\end{document}
